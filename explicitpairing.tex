\documentclass[12pt]{article}
\usepackage{setspace}
\usepackage{amsmath}
\usepackage{amsfonts}
\usepackage{amssymb}
\usepackage{amsthm}
\usepackage{mathrsfs}
\usepackage{amsxtra}
\usepackage[all,cmtip]{xy}
\usepackage{graphicx}
\usepackage{color}
\usepackage{comment}
\usepackage{verbatim}
\usepackage{hyperref}

\newtheorem{thm}{Theorem}[section]

\newtheorem{fact}{Fact}[thm]
\newtheorem{lem}[thm]{Lemma}
\newtheorem{prop}[thm]{Proposition}
\newtheorem*{conj}{Conjecture}
\newtheorem{cor}[thm]{Corollary}
\newtheorem*{thmA}{Theorem A}
\newtheorem*{thmB}{Theorem B}
\newtheorem*{thmC}{Theorem C}


\theoremstyle{definition}
\newtheorem{defn}[thm]{Definition}
\newtheorem{rem}[thm]{Remark}
\newtheorem*{claim}{Claim}
\newtheorem{exam}[thm]{Example}


\def\Xint#1{\mathchoice
{\XXint\displaystyle\textstyle{#1}}%
{\XXint\textstyle\scriptstyle{#1}}%
{\XXint\scriptstyle\scriptscriptstyle{#1}}%
{\XXint\scriptscriptstyle\scriptscriptstyle{#1}}%
\!\int}
\def\XXint#1#2#3{{\setbox0=\hbox{$#1{#2#3}{\int}$}
\vcenter{\hbox{$#2#3$}}\kern-.5\wd0}}
\def\multint{\Xint\times}
\def\dashint{\Xint-}

\def\image{\text{Im}}
\def\And{\quad\hbox{ and }\quad}
\def\cc{completely continuous\ }
\def\ccly{completely continuously\ }
\def\fg{{\frak g}}
\def\fn{{\frak n}}
\def\ft{{\frak t}}
\def\fb{{\frak b}}
\def\fU{{\frak U}}
\def\Gal{\operatorname{Gal}}

\def\lra{\longrightarrow}

\def\bA{\mathbb{A}}
\def\cA{{\mathcal A}}
\def\fA{{\frak A}}
\def\cB{{\mathcal B}}
\def\cC{{\mathcal C}}
\def\tcC{{\widetilde{\cal C}}}
\def\cD{{\mathcal D}}
\def\Dist{\hbox{Dist}}
\def\cE{{\mathcal E}}
\def\fD{{\frak D}}
\def\cR{{\mathcal R}}
\def\tR{{\frak R}}
\def\tF{{\frak F}}
\def\cG{\mathcal{G}}
\def\bG{\mathbb{G}}
\def\tJ{{\frak J}}
\def\tC{{\frak C}}
\def\cF{{\mathcal F}}
\def\cK{{\mathcal K}}
\def\tK{{\widetilde{\cal K}}}
\def\cS{{\mathcal S}}
\def\dV{{(V^\ast)}}
\def\fI{{\frak I}}
\def\fS{\frak S}
\def\cT{\mathcal{T}}
\def\fp{\frak p}
\def\cP{\mathcal{P}}
\def\fP{\frak P}
\def\fH{{\frak H}}
\def\cL{{\mathcal L}}
\def\cM{{\mathcal M}}
\def\cO{\mathcal{O}}
\def\cU{\mathcal{U}}
\def\cV{\mathcal{V}}
\def\bX{\mathbb{X}}
\def\cX{{\mathcal X}}
\def\Z{{\Bbb Z}}
\def\R{{\Bbb R}}
\def\B{{\mathbb{B}}}
\def\bP{{\Bbb P}}
\def\D{{\Bbb D}}
\def\A{{\Bbb A}}
\def\C{{\Bbb C}}
\def\N{{\Bbb N}}
\def\Q{{\Bbb Q}}
\def\F{{\Bbb F}}
\def\P{\mathbb{P}}

\def\inv{^{-1}}
\def\End{\text{End}}
\def\Hom{\text{Hom}}
\def\supp{\text{supp}\,}
\def\Div{\operatorname{Div}}
\def\ord{\text{ord}\,}
\def\det{\text{det}\,}
\def\Step{\text{Step}\,}
\def\trunc{\text{Trunc}\,}
\def\ker{\text{ker}}
\def\Ker{\text{Ker}}
\def \oQ {{\overline\Q}}
\def \fQ {{\frak Q}}
\def \OoQ {{\cal O}_{\oQ_p}}
\def\bx {{\bf x}}
\def\by{{\bf y}}
\def\Tr{\operatorname{Tr}}
\def\GL{\operatorname{GL}}
\def\SL{\operatorname{SL}}
\def\parx{\partial_X}
\def\pary{\partial_Y}
\def\dlog{\operatorname{dlog}}
\def\Sym{\operatorname{Sym}}
\def\Aut{\operatorname{Aut}}
\def\dist{\widetilde{\cD}_{poly}}
\def\Symb{\operatorname{Symb}}
\def\Stab{\operatorname{Stab}}
\def\Meas{\operatorname{Meas}}
\def\Real{\operatorname{Re}}
\def\bAV{\mathbb{A}_{V,f}}
\def\GL{\operatorname{GL}}
\def\Olog{\Omega_{\text{log}}}
\def\sign{\operatorname{sign}}
\def\Res{\operatorname{Res}}
\def\hill{\sigma_{\text{Hill}}}
\def\Supp{\operatorname{Support}}
\def\bs{\mathbf{s}}
\def\bx{\mathbf{x}}
\def\br{\mathbf{r}}
\def\bv{\mathbf{v}}
\def\bw{\mathbf{w}}
\def\cW{\mathcal{W}}
\def\F{{\Bbb F}}
\def\P{\mathbb{P}}
\def\cP{\mathcal{P}}
\def\Norm{\operatorname{N}}
\def\LC{\mathcal{LC}}
\def\LP{\mathcal{LP}}
\def\LA{\mathcal{LA}}
\def\Symm{\operatorname{Sym}}
\def\Funct{\operatorname{Funct}}
\def\res{\operatorname{res}}
\def\sgn{\operatorname{sgn}}
\def\Coeff{\operatorname{Coeff}}
\def\Span{\operatorname{Span}}
\def\Frac{\operatorname{Frac}}
\def\cH{\mathcal{H}}
\def\cZ{\mathcal{Z}}
\def\bD{\mathbf{D}}

\def\PGL{\operatorname{PGL}}
\def\wt{\operatorname{wt}}
\def\ev{\operatorname{ev}}
\def\fpv{f^{\prime\vee}}
\def\CoInd{\operatorname{CoInd}}

\newcommand{\insertpic}[2][0.4]{\begin{center}\includegraphics*[scale=0.30]{#2}\end{center}}
\topmargin -0.4in
\textwidth 6.5in
\textheight 9 in

\hoffset=-.55in
\voffset=-0.05in
\begin{document}
\title{Explicit computations of cup products of modular symbols}
\maketitle

Given a modular symbol $\Phi\in\Symb_\Gamma(M)$, we have $\sigma_\Phi \in H^1_c(\Gamma,M)$ (group actions?) given by
$$\sigma_\Phi(\alpha) :=\Phi\{r,\alpha r\}$$
for some choice of cusp $r\in \P^1(\Q)$. Note that different choices of cusps produce cohomologous cocycles.
Note: It may be easier to use the homogeneous $1$-cocycle $\widetilde{\sigma}_{\Phi}(\alpha,\beta):=\Phi\{\alpha r,\beta r\}$

The image of $\Symb_\Gamma(M)$ in $H^1(\Gamma,M)$ is the parabolic cohomology $H^1_p(\Gamma,M)$. The cup product  $H^1(\Gamma,M)\times H^1(\Gamma,N)\rightarrow H^2(\Gamma,M\otimes N)$ can be described on inhomogeneous cochains $f,g$by
$$ f\cup g (\alpha,\beta):=f(\alpha)\otimes g(\alpha\beta)$$
The homogeneous $2$-cocycle is given by
$$ f\cup g (\alpha,\beta,\gamma):=f(\alpha,\beta) \otimes g(\beta,\gamma)$$

Thus
$$\widetilde{\sigma}_\Phi \cup \widetilde{\sigma}_\Psi(\alpha,\beta,\gamma)= \Phi\{\alpha r,\beta r\}\otimes\Psi\{\beta r,\gamma r\}$$

There is also a cup product defined on compactly supported cohomology, $H^1_c(\Gamma,M)\times H^1_c(\Gamma,N)\rightarrow H^2_c(\Gamma,M\otimes N)$. According to Hida \ref{}, $H^2_c(\Gamma, M')=H^2_P(\Gamma,M')=Z^2(\Gamma,M')/B_P^2(\Gamma,M')$. [insert explicit description]

\begin{prop}
Suppose given a $\Z[\Gamma]$ homomorphism $M\otimes N\rightarrow \Q$. The cup product induces a pairing $[~,~] : \Symb_\Gamma(M)\times \Symb_\Gamma(N) \rightarrow \Q$ by
$$[\Phi,\Psi]=\sum_{\langle \pi\rangle } u(\pi)$$
where $\partial u= \sigma_\Phi \cup \sigma_\Psi$ and $\langle \pi\rangle $ runs over distinct conjugacy classes of parabolic subgroups.
\end{prop}

\section{The case of $\SL_2(\Z)$}
%Let $M$ be a (finite dimensional?) $\Q_p$ vector space with a (left?) action of $\SL_2(\Z)$. We wish to compute, explicitly, the e

\section{Congruence subgroups}

Now we use the results of the previous section to compute the pairing for the congruence subgroups $\Gamma_0(N)$. By Shapiro's lemma, we have
	$$H^2(\Gamma_0(N),\Q)=H^2(\Gamma_0(1),\CoInd^{\Gamma_0(1)}_{\Gamma_0(N)}\Q)$$
Here $\CoInd^{\Gamma_0(N)}_{\Gamma_0(1)}\Q=\Hom_{\Z[\Gamma_0(N)]}(\Z[\Gamma_0(1)],\Q)$.

The description of parabolic cochains is a little more subtle when we have a non-trivial action. $H^2_P(G,M):= Z^2(G,M)/B^2_P(G,M)$, where $B_P(G, M)=\{ \delta u : u\in (\gamma-1)\cdot M \text{ for all } \gamma\in P\}$

But we don't even need that. Our strategy is to produce a $1$-chain in $Z^1(\Gamma_0(1),\CoInd)$, then trace down to $Z^1(\Gamma_0(N),\Q)$ and compute the isomorphism there. 

Let $\sigma$ be a $4$-torsion element and $\tau$ a $6$-torsion element. $\sigma=S=\begin{pmatrix} 0 & -1\\ 1 & 0\end{pmatrix}$ and $\tau=ST=\begin{pmatrix} 0 & -1 \\ 1 & 1\end{pmatrix}$


\begin{equation*}
	\delta u(1,1)=u(1), \text{ and } \delta u(-1,-1) =u(-1)+(-1\cdot u(-1))-u(1)=2u(-1)-u(1)
\end{equation*}
%In our application, $u(1,1)=0$ and $u(-1,-1)=0$, so we can conclude $u(1)=u(-1)=0$. Actually, I don't know this for certain. In fact, this may not be true for the induced cocycle. But in any case, we know those terms from the above equations. 

\begin{align*}
\delta u(\sigma,\sigma) &=u(\sigma)+\sigma\cdot u(\sigma)-u(-1)\\
\end{align*}

\begin{align*}
\delta u (\tau,\tau) & = u(\tau) +\tau\cdot u(\tau) - u(\tau^2)\\
\delta u(\tau^2,\tau) & = u(\tau^2) +\tau^2\cdot u(\tau) - u(-1)\\
\tau\cdot \delta u(\tau^2,\tau) &= \tau\cdot u(\tau^2) + u(\tau) -\tau\cdot u(-1)\\
\delta u(\tau,\tau^2) & = u(\tau)+\tau\cdot u(\tau^2)- u(-1)
\end{align*}

Each value is an element of $\CoInd$, which is a finite dimensional $\Q$-vector space. For example, when $\Gamma=\Gamma_0(p)$, $\CoInd^{\Gamma_0(1)}_{\Gamma_0(p)}(\Q)$ is $p+1$ dimensional. 
$$\tau\cdot( \delta u(\tau,\tau^2) -u(-1)) = \tau\cdot u(\tau)+\tau^2 \cdot u(\tau^2)$$
$$\tau^{2} \cdot(\delta u(\tau,\tau^2) - u(-1)) =  \tau^{2}\cdot u(\tau) +u(\tau^2)$$







\end{document}