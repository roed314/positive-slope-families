\documentclass[12pt]{article}
\usepackage{setspace}
\usepackage{amsmath}
\usepackage{amsfonts}
\usepackage{amssymb}
\usepackage{amsthm}
\usepackage{mathrsfs}
\usepackage{amsxtra}
\usepackage[all,cmtip]{xy}
\usepackage{graphicx}
\usepackage{color}
\usepackage{comment}
\usepackage{verbatim}
\usepackage{hyperref}

\newtheorem{thm}{Theorem}[section]

\newtheorem{fact}{Fact}[thm]
\newtheorem{lem}[thm]{Lemma}
\newtheorem{prop}[thm]{Proposition}
\newtheorem*{conj}{Conjecture}
\newtheorem{cor}[thm]{Corollary}
\newtheorem*{thmA}{Theorem A}
\newtheorem*{thmB}{Theorem B}
\newtheorem*{thmC}{Theorem C}


\theoremstyle{definition}
\newtheorem{defn}[thm]{Definition}
\newtheorem{rem}[thm]{Remark}
\newtheorem*{claim}{Claim}
\newtheorem{exam}[thm]{Example}


\def\Xint#1{\mathchoice
{\XXint\displaystyle\textstyle{#1}}%
{\XXint\textstyle\scriptstyle{#1}}%
{\XXint\scriptstyle\scriptscriptstyle{#1}}%
{\XXint\scriptscriptstyle\scriptscriptstyle{#1}}%
\!\int}
\def\XXint#1#2#3{{\setbox0=\hbox{$#1{#2#3}{\int}$}
\vcenter{\hbox{$#2#3$}}\kern-.5\wd0}}
\def\multint{\Xint\times}
\def\dashint{\Xint-}

\def\image{\text{Im}}
\def\And{\quad\hbox{ and }\quad}
\def\cc{completely continuous\ }
\def\ccly{completely continuously\ }
\def\fg{{\frak g}}
\def\fn{{\frak n}}
\def\ft{{\frak t}}
\def\fb{{\frak b}}
\def\fU{{\frak U}}
\def\Gal{\operatorname{Gal}}

\def\lra{\longrightarrow}

\def\bA{\mathbb{A}}
\def\cA{{\mathcal A}}
\def\fA{{\frak A}}
\def\cB{{\mathcal B}}
\def\cC{{\mathcal C}}
\def\tcC{{\widetilde{\cal C}}}
\def\cD{{\mathcal D}}
\def\Dist{\hbox{Dist}}
\def\cE{{\mathcal E}}
\def\fD{{\frak D}}
\def\cR{{\mathcal R}}
\def\tR{{\frak R}}
\def\tF{{\frak F}}
\def\cG{\mathcal{G}}
\def\bG{\mathbb{G}}
\def\tJ{{\frak J}}
\def\tC{{\frak C}}
\def\cF{{\mathcal F}}
\def\cK{{\mathcal K}}
\def\tK{{\widetilde{\cal K}}}
\def\cS{{\mathcal S}}
\def\dV{{(V^\ast)}}
\def\fI{{\frak I}}
\def\fS{\frak S}
\def\cT{\mathcal{T}}
\def\fp{\frak p}
\def\cP{\mathcal{P}}
\def\fP{\frak P}
\def\fH{{\frak H}}
\def\cL{{\mathcal L}}
\def\cM{{\mathcal M}}
\def\cO{\mathcal{O}}
\def\cU{\mathcal{U}}
\def\cV{\mathcal{V}}
\def\bX{\mathbb{X}}
\def\cX{{\mathcal X}}
\def\Z{{\Bbb Z}}
\def\R{{\Bbb R}}
\def\B{{\mathbb{B}}}
\def\bP{{\Bbb P}}
\def\D{{\Bbb D}}
\def\A{{\Bbb A}}
\def\C{{\Bbb C}}
\def\N{{\Bbb N}}
\def\Q{{\Bbb Q}}
\def\F{{\Bbb F}}
\def\P{\mathbb{P}}

\def\inv{^{-1}}
\def\End{\text{End}}
\def\Hom{\text{Hom}}
\def\supp{\text{supp}\,}
\def\Div{\operatorname{Div}}
\def\ord{\text{ord}\,}
\def\det{\text{det}\,}
\def\Step{\text{Step}\,}
\def\trunc{\text{Trunc}\,}
\def\ker{\text{ker}}
\def\Ker{\text{Ker}}
\def \oQ {{\overline\Q}}
\def \fQ {{\frak Q}}
\def \OoQ {{\cal O}_{\oQ_p}}
\def\bx {{\bf x}}
\def\by{{\bf y}}
\def\Tr{\operatorname{Tr}}
\def\GL{\operatorname{GL}}
\def\SL{\operatorname{SL}}
\def\parx{\partial_X}
\def\pary{\partial_Y}
\def\dlog{\operatorname{dlog}}
\def\Sym{\operatorname{Sym}}
\def\Aut{\operatorname{Aut}}
\def\dist{\widetilde{\cD}_{poly}}
\def\Symb{\operatorname{Symb}}
\def\Stab{\operatorname{Stab}}
\def\Meas{\operatorname{Meas}}
\def\Real{\operatorname{Re}}
\def\bAV{\mathbb{A}_{V,f}}
\def\GL{\operatorname{GL}}
\def\Olog{\Omega_{\text{log}}}
\def\sign{\operatorname{sign}}
\def\Res{\operatorname{Res}}
\def\hill{\sigma_{\text{Hill}}}
\def\Supp{\operatorname{Support}}
\def\bs{\mathbf{s}}
\def\bx{\mathbf{x}}
\def\br{\mathbf{r}}
\def\bv{\mathbf{v}}
\def\bw{\mathbf{w}}
\def\cW{\mathcal{W}}
\def\F{{\Bbb F}}
\def\P{\mathbb{P}}
\def\cP{\mathcal{P}}
\def\Norm{\operatorname{N}}
\def\LC{\mathcal{LC}}
\def\LP{\mathcal{LP}}
\def\LA{\mathcal{LA}}
\def\Symm{\operatorname{Sym}}
\def\Funct{\operatorname{Funct}}
\def\res{\operatorname{res}}
\def\sgn{\operatorname{sgn}}
\def\Coeff{\operatorname{Coeff}}
\def\Span{\operatorname{Span}}
\def\Frac{\operatorname{Frac}}
\def\cH{\mathcal{H}}
\def\cZ{\mathcal{Z}}
\def\bD{\mathbf{D}}

\def\PGL{\operatorname{PGL}}
\def\wt{\operatorname{wt}}
\def\ev{\operatorname{ev}}
\def\fpv{f^{\prime\vee}}
\def\CoInd{\operatorname{CoInd}}

\newcommand{\insertpic}[2][0.4]{\begin{center}\includegraphics*[scale=0.30]{#2}\end{center}}
\topmargin -0.4in
\textwidth 6.5in
\textheight 9 in

\hoffset=-.55in
\voffset=-0.05in
\begin{document}
\title{Explicit computations of cup products of modular symbols}

\maketitle


Given a modular symbol $\Phi\in\Symb_\Gamma(M)$, we have $\sigma_\Phi \in H^1_c(\Gamma,M)$ (group actions?) given by
$$\sigma_\Phi(\alpha) :=\Phi\{r,\alpha r\}$$
for some choice of cusp $r\in \P^1(\Q)$. Note that different choices of cusps produce cohomologous cocycles.
Note: It may be easier to use the homogeneous $1$-cocycle $\widetilde{\sigma}_{\Phi}(\alpha,\beta):=\Phi\{\alpha r,\beta r\}$

The image of $\Symb_\Gamma(M)$ in $H^1(\Gamma,M)$ is the parabolic cohomology $H^1_p(\Gamma,M)$. The cup product  $H^1(\Gamma,M)\times H^1(\Gamma,N)\rightarrow H^2(\Gamma,M\otimes N)$ can be described on inhomogeneous cochains $f,g$by
$$ f\cup g (\alpha,\beta):=f(\alpha)\otimes g(\alpha\beta)$$
The homogeneous $2$-cocycle is given by
$$ f\cup g (\alpha,\beta,\gamma):=f(\alpha,\beta) \otimes g(\beta,\gamma)$$

Thus
$$\widetilde{\sigma}_\Phi \cup \widetilde{\sigma}_\Psi(\alpha,\beta,\gamma)= \Phi\{\alpha r,\beta r\}\otimes\Psi\{\beta r,\gamma r\}$$

There is also a cup product defined on compactly supported cohomology, $H^1_c(\Gamma,M)\times H^1_c(\Gamma,N)\rightarrow H^2_c(\Gamma,M\otimes N)$. According to Hida \ref{}, $H^2_c(\Gamma, M')=H^2_P(\Gamma,M')=Z^2(\Gamma,M')/B_P^2(\Gamma,M')$, where
$$
B^2_P = \{\partial u | u\in C^1(\Gamma, M') \text{ with } u(\pi)\in (\pi-1)M' \text{ for all } \pi\in P\}
$$
and $P$ is the union of the stabilizers in $\Gamma$ of the cusps [pg 347 of Hida].
\begin{prop}
Suppose given a $\Z[\Gamma]$ homomorphism $M\otimes N\rightarrow \Q$. The cup product induces a pairing $[~,~] : \Symb_\Gamma(M)\times \Symb_\Gamma(N) \rightarrow \Q$ by
$$[\Phi,\Psi]=\sum_{\langle \pi\rangle } u(\pi)$$
where $\partial u= \sigma_\Phi \cup \sigma_\Psi$ and $\langle \pi\rangle $ runs over distinct conjugacy classes of parabolic subgroups. [Are we sure about this? Hida describes $P$ differently
\end{prop}

\section{The case of $\SL_2(\Z)$}
%Let $M$ be a (finite dimensional?) $\Q_p$ vector space with a (left?) action of $\SL_2(\Z)$. We wish to compute, explicitly, the e
\begin{equation*}
	\delta u(1,1)=u(1), \text{ and } \delta u(-1,-1) =u(-1)+u(-1)-u(1)=2u(-1)-u(1)
\end{equation*}
%In our application, $u(1,1)=0$ and $u(-1,-1)=0$, so we can conclude $u(1)=u(-1)=0$. Actually, I don't know this for certain. In fact, this may not be true for the induced cocycle. But in any case, we know those terms from the above equations. 

\begin{align*}
\delta u(\sigma,\sigma) &=u(\sigma)+ u(\sigma)-u(-1)\\
\end{align*}

\begin{align*}
\delta u (\tau,\tau) & = u(\tau) + u(\tau) - u(\tau^2)\\
\delta u(\tau,\tau^2) & = u(\tau)+u(\tau^2)- u(-1)
\end{align*}


\section{Congruence subgroups}
We can use the Corestriction map and the results of the previous section to compute the pairing on congruence subgroups.
%Now we use the results of the previous section to compute the pairing for the congruence subgroups $\Gamma_0(N)$. By Shapiro's lemma, we have
%	$$H^2(\Gamma_0(N),\Q)=H^2(\Gamma_0(1),\CoInd^{\Gamma_0(1)}_{\Gamma_0(N)}\Q)$$
%Here $\CoInd^{\Gamma_0(N)}_{\Gamma_0(1)}\Q=\Hom_{\Z[\Gamma_0(N)]}(\Z[\Gamma_0(1)],\Q)$.
%
%The description of parabolic cochains is a little more subtle when we have a non-trivial action. $H^2_P(G,M):= Z^2(G,M)/B^2_P(G,M)$, where $B_P(G, M)=\{ \delta u : u\in (\gamma-1)\cdot M \text{ for all } \gamma\in P\}$
%
%But we don't even need that. Our strategy is to produce a $1$-chain in $Z^1(\Gamma_0(1),\CoInd)$, then trace down to $Z^1(\Gamma_0(N),\Q)$ and compute the isomorphism there. 
%
%Let $\sigma$ be a $4$-torsion element and $\tau$ a $6$-torsion element. $\sigma=S=\begin{pmatrix} 0 & -1\\ 1 & 0\end{pmatrix}$ and $\tau=ST=\begin{pmatrix} 0 & -1 \\ 1 & 1\end{pmatrix}$
%
%
%\begin{equation*}
%	\delta u(1,1)=u(1), \text{ and } \delta u(-1,-1) =u(-1)+(-1\cdot u(-1))-u(1)=2u(-1)-u(1)
%\end{equation*}
%%In our application, $u(1,1)=0$ and $u(-1,-1)=0$, so we can conclude $u(1)=u(-1)=0$. Actually, I don't know this for certain. In fact, this may not be true for the induced cocycle. But in any case, we know those terms from the above equations. 
%
%\begin{align*}
%\delta u(\sigma,\sigma) &=u(\sigma)+\sigma\cdot u(\sigma)-u(-1)\\
%\end{align*}
%
%\begin{align*}
%\delta u (\tau,\tau) & = u(\tau) +\tau\cdot u(\tau) - u(\tau^2)\\
%\delta u(\tau^2,\tau) & = u(\tau^2) +\tau^2\cdot u(\tau) - u(-1)\\
%\tau\cdot \delta u(\tau^2,\tau) &= \tau\cdot u(\tau^2) + u(\tau) -\tau\cdot u(-1)\\
%\delta u(\tau,\tau^2) & = u(\tau)+\tau\cdot u(\tau^2)- u(-1)
%\end{align*}
%
%Each value is an element of $\CoInd$, which is a finite dimensional $\Q$-vector space. For example, when $\Gamma=\Gamma_0(p)$, $\CoInd^{\Gamma_0(1)}_{\Gamma_0(p)}(\Q)$ is $p+1$ dimensional. 
%$$\tau\cdot( \delta u(\tau,\tau^2) -u(-1)) = \tau\cdot u(\tau)+\tau^2 \cdot u(\tau^2)$$
%$$\tau^{2} \cdot(\delta u(\tau,\tau^2) - u(-1)) =  \tau^{2}\cdot u(\tau) +u(\tau^2)$$
%

\section{Pairings on coefficient modules}

We have assumed given a $\Gamma$-equivariant pairing $\langle~,~ \rangle : M\times M \longrightarrow \Q_p$ (or another trial $\Gamma$ representation). We explicitly write down the pairing in the cases of interest and check that the necessary equivariance is satisfied. 

\subsection{$\Sym^k(\Q_p^2)$}

We identify $\Sym^k(\Q_p^2)$ with $\Q_p[X,Y]_k$, and endow it with left action of $\GL_2(\Q)$ by $\gamma \cdot P(X,Y) = P((X,Y)\gamma)=P(aX+cY,bX+dY).$ 
\begin{defn}
Define a bilinear pairing on $\Sym^k(\Q_p^2)$ by
\begin{equation*}
	\langle X^iY^{k-i}, X^j Y^{k-j} \rangle = \begin{cases} (-1)^i {k \choose i} & \text{ if } i+j=k\\
												0	& \text{ otherwise} \end{cases}
\end{equation*}
and extending via linearity.
\end{defn}


\begin{lem}
For all $\gamma\in \GL_2(\Q)$, $P,Q\in\Sym^k(\Q_p^2)$, we have $\langle \gamma\cdot P,\gamma \cdot Q\rangle = \det(\gamma)^k \langle P,Q\rangle$.
\end{lem}
\begin{proof}
This can be seen via an ugly computation. (There should be an easy way to see this in terms of the wedge product).
\end{proof}

For the purposes of Hecke-equivariance, we define a twisted pairing using the Atkin-Lehner involution $W_N = \begin{pmatrix} 0 & -1 \\ N & 0 \end{pmatrix}$.
\begin{defn}
The twisted pairing on $\Sym^k(\Q_p^2)$ is defined by
\begin{equation}
	\langle P, Q\rangle^t := \langle P, W_N \cdot Q\rangle.
\end{equation}
\end{defn}
More explicitly,
\begin{equation*}
	\langle X^iY^{k-i}, X^j Y^{k-j} \rangle_N^t = \langle X^i Y^{k-i}, (-Y)^j(NX)^{k-j}\rangle= \begin{cases} (-1)^k N^i {k \choose i} & \text{ if } i=j\\
												0	& \text{ otherwise} \end{cases}
\end{equation*}


\section{$\cD_k$}
We can view $\Sym^k(\Q_p^2)$ as a quotient of $\cD_k$, and the pairings from the previous section extends uniquely in a $\Sigma_0(p)$-equivariant way to $\cD_k$. Moreover, the twisted pairing behaves well with respect to our finite approximations. [Insert details]

\begin{defn}
Let $\langle~,~\rangle^t_N : \cD_k\times\cD_k\longrightarrow \Q_p$ be the $\Q_p$-bilinear pairing
\begin{equation}
	\langle \mu, \nu \rangle_N^t := \sum_{n=0}^\infty {k \choose n} (Np)^n \mu(z^n)\nu(z^n)
\end{equation}
\end{defn}


\section{$\cD(\Lambda)$}

\begin{defn}
Let $\langle ~, ~\rangle^t_N :\cD(\Lambda)\times\cD(\Lambda) \longrightarrow \Lambda$ be the $\Lambda$-bilinear pairing
\begin{equation}
	\langle \mu, \nu \rangle_N^t = \sum_{n=0}^\infty {\frac{\log_p(1+pw)}{\log_p(1+p)} \choose n} (Np)^n \mu(z^n)\nu(z^n).
\end{equation}
\end{defn}

\begin{lem}
The above pairing interpolates the weight-$k$ twisted pairings, and satisfies the following equivariance property: for all $\gamma \Gamma_0(p)$, $\mu,\nu\in \cD(\Lambda)$, 
\begin{equation}
	\langle \mu | \gamma, \nu\rangle_N^t  = \langle \mu, \nu| W_{Np}^{-1}\gamma^* W_{Np}\rangle_N^t.
\end{equation}	
\end{lem}

\end{document}