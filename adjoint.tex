\documentclass[10pt]{amsart}

\title{Computing Adjoint $L$-functions}
\author{David Roe}
\address{Department of Mathematics, University of Pittsburgh, 301 Thackeray Hall, Pittsburgh, PA , United States, 15260.}
\email{roed.math@gmail.com}
\author{Ander Steele}
\address{Department of Mathematics, University of California Santa Cruz, 1156 High Street, Santa Cruz, CA, United States, 95060}
\email{gasteele@ucsc.edu}

%\subjclass[2010]{}
%\keywords{}

\usepackage{amsmath}
\usepackage{amssymb}
\usepackage{amsrefs}
% Fonts
\usepackage{mathrsfs}
% Enumitem
\usepackage{enumitem}
% Hyperrefs
\usepackage{hyperref}

%\usepackage{tikz}
%\usetikzlibrary{shapes,arrows,calc,matrix}
%\usepackage{tikz-cd}

%%%%% for spacing
\usepackage{lipsum}
\usepackage{setspace}

%%%%%%%%%%%%%%% THEOREM STYLES %%%%%%%%%%%%%%%
\theoremstyle{plain}
      \newtheorem{theorem}{Theorem}[section]
      \newtheorem*{theorem*}{Theorem}
      \newtheorem{proposition}[theorem]{Proposition}
      \newtheorem{lemma}[theorem]{Lemma}
      \newtheorem{corollary}[theorem]{Corollary}

      \theoremstyle{definition}
      \newtheorem{definition}[theorem]{Definition}

      %\theoremstyle{remark}
      \newtheorem{remark}[theorem]{Remark}
      \newtheorem{example}[theorem]{Example}
      
      \newtheorem{conjecture}[theorem]{Conjecture}
%%%%%%%%%%%%%%% RINGS AND GROUPS %%%%%%%%%%%%%%%
\newcommand{\ZZ}{{\mathbb{Z}}}
\newcommand{\NN}{{\mathbb{N}}}
\newcommand{\CC}{{\mathbb{C}}}
\newcommand{\QQ}{{\mathbb{Q}}}
\newcommand{\RR}{{\mathbb{R}}}
\newcommand{\OK}{\mathcal{O}_K}
\newcommand{\cH}{\mathcal{H}}
\newcommand{\cA}{\mathcal{A}}
\newcommand{\cD}{\mathcal{D}}


%%%%%%%%%%%%%%% NAMED OPERATORS %%%%%%%%%%%%%%%
\DeclareMathOperator{\Gal}{Gal}
\DeclareMathOperator{\Hom}{Hom}
\DeclareMathOperator{\ord}{ord}

\DeclareMathOperator{\GL}{GL}
\DeclareMathOperator{\SL}{SL}
\DeclareMathOperator{\PGL}{PGL}
\DeclareMathOperator{\Sym}{Sym}
\DeclareMathOperator{\Symb}{Symb}
\DeclareMathOperator{\BSymb}{BSymb}
\DeclareMathOperator{\Hh}{H}
\DeclareMathOperator{\cor}{cor}
\DeclareMathOperator{\ad}{ad}
\DeclareMathOperator{\Sp}{Sp}



%%%%%%%%%%%% BEGIN DOCUMENT %%%%%%%%%%%
\usepackage{todonotes}

\begin{document}
\maketitle

\section{Introduction}

\section{Review}

Let $E$ be a complete, discrete valued field.  \todo{think about what works in characteristic $p$}

\section{Cohomology of Modular Curves} \label{sec:mod_curve_cohom}
% Reference appendix to Hida's Elementary Theory of L-functions and Eisenstein Series
%The theory of modular symbols allows us to efficiently compute with spaces of modular forms. We briefly recall 

[Short summary: We compute modular forms through modular symbols. Modular symbols are $H^1_c$, $H^1_p$ is a natural quotient., and our pairings factor through this projection. Thus, all our computations reduce to group cohomology]



Let $\Gamma$ be a congruence subgroup---in our examples, we will take $\Gamma_0(N)$. Denote by $\Delta_0$ the degree-$0$ divisors on $\mathbb{P}^1(\QQ)$, equipped with the left action of $\Gamma$ by fractional-linear-transformations. Let $M$ be any right $\ZZ[\Gamma]$-module. The space of $M$-valued modular symbols for $\Gamma$ is defined as
\begin{equation*}
	\Symb_\Gamma(M) := \Hom_{\ZZ[\Gamma]}(\Delta_0, M).
\end{equation*}

\begin{proposition}[Ash-Stevens]
Suppose $\Gamma$ acts invertibly on $\cH$, or $6$ acts invertibly on $M$. Then $H^1_c(\Gamma,M)$ is [canonically and functorially?] isomorphic to $\Symb_\Gamma(M)$.
%and we have an exact sequence:
%\begin{equation*}
%\BSymb_\Gamma(M)\longrightarrow \Symb_\Gamma(M)\longrightarrow H^1_p(\Gamma, M)\longrightarrow 0
%\end{equation*}
\end{proposition}

The compactly supported cohomology of the group $\Gamma$ is defined as \todo{do we need this?}
[Need to distinguish the cohomology of the modular curve and group cohomology, and make precise comparisons... eg. $H^*(\Gamma, M) = H^*(Y, \widetilde{M})$]

%the the compactly supported cohomology of the open modular curve $\Gamma\cH$, with coefficients in the local system blah blah blah.


We have a natural map $H^1_c(\Gamma, M)\longrightarrow H^1(\Gamma,M)$, and thus a natural map from $\Symb_\Gamma(M)\longrightarrow H^1(\Gamma,M)$. Concretely, a modular symbol $\Phi\in \Symb_\Gamma(M)$ is mapped to the cohomology class of the cocycle $\sigma_\Phi$ defined by
\begin{align*}
\sigma_{\Phi}(\alpha, \beta) &= \Phi(\{\alpha \cdot \infty, \beta \cdot \infty\}).% \\
%\sigma_{\Psi}(\alpha, \beta) &= \Psi(\{\alpha \cdot \infty, \beta \cdot \infty\}).
\end{align*}
\begin{definition}
The parabolic cohomology $H^1_p(\Gamma,M)$ is defined to be the image of the natural map $H^1_c(\Gamma,M)\longrightarrow H^1_p(\Gamma,M)$.
\end{definition}
The kernel of the map $\Symb_\Gamma(M)\longrightarrow H^1_p(\Gamma,M)$ is the subspace of boundary symbols, which correspond to Eisenstein series. See [Bellaiche reference on Eichler-Shimura]



\subsection{Classical Modular Symbols}
\begin{definition}
Let $F$ be a field and $k\geq 0$ be an integer. Define $V_k(F):=\Sym^k(F^2)$. We identify $V_k(F)$ with homogeneous degree-$k$ polynomials in two variables, $V_k(F) = F[X,Y]_k$. We equip $V_k(F)$ with a right-action of $\GL_2(F)$ by $P(X,Y)|\gamma =\det(\gamma)^{k}P((X,Y)|\gamma^{-1})=P(dX-cY,-bX+aY)$, where $\gamma=\begin{pmatrix} a & b \\ c & d\end{pmatrix}$. 
\end{definition}
The connection with modular forms is described by the Eichler-Shimura Isomorphism. A cuspidal modular form $f\in S_{k+2}(\Gamma)$ gives rise to a modular symbol $\phi_{f}\in \Symb_{\Gamma} (V_k(\CC))$ by
\begin{equation*}
	\phi_{f} \{r, s\} = \int_r^s f(z) (zX+Y)^k dz.
\end{equation*}
Let $\iota = \begin{pmatrix} -1 & 0 \\ 0 & 1\end{pmatrix}$, which normalizes $\Gamma$ and hence induces an involution on $H^1_p(\Gamma, V_k(\CC))$ [insert reference] (this involution roughly corresponds to the action of complex conjugation). Thus, $H^1_p(\Gamma, V_k(\CC))$ decomposes into $H^1_p(\Gamma,V_k(\CC))^+\oplus H^1_p(\Gamma,V_k(\CC))^-$, and $\phi_f = \phi_f^+ + \phi_f^-$. 

For each cusp form $f\in S_{k+2}(\Gamma)$, a theorem of Shimura [need ref!] gives complex periods $\Omega_f^\pm$ such that $\phi_f^\pm/\Omega_f^\pm \in \Symb_\Gamma(V_k(\overline{L}))$. Here $L$ is a number field containing the coefficients of the regularized $q$-expansion of $f$.
%see 3.4 of Pollack-Stevens
\begin{theorem}[Eichler-Shimura]
Let $F$ be a number field. The maps $S_{k+2}(\Gamma,F)\longrightarrow H^1_p(\Gamma, V_k(F))^\pm$, $f\mapsto \phi_{f}^\pm$ are isomorphisms of Hecke modules. That is,
\begin{equation*}
	H^1_p(\Gamma, V_k(F)) \cong S_{k+2}(\Gamma,F)\oplus \overline{S_{k+2}(\Gamma,F)}
\end{equation*}
\end{theorem}

\subsection{Overconvergent Modular Symbols}
By replacing the vector spaces $V_k(F)$ with spaces of $p$-adic distributions $\cD_k(F)$, Stevens \cite{Stevens} constructs \emph{overconvergent} modular symbols corresponding to  overconvergent modular forms.

We closely follow \cite{Bellaiche}, referring the reader there for details. Let $L$ be a commutative, noetherian $\QQ_p$-Banach algebra with norm $|\cdot |$. 

For each $r\in p^\QQ$, denote by $\cA[r](L)$ the $L$-module of functions $f:\ZZ_p\rightarrow L$ which, for all $e\in\ZZ_p$,  have power series expansions 
\begin{equation*}
	f(z) = \sum_{n \geq 0 } a_n(e) (z-e)^n
\end{equation*}
converging on the closed balls $B[e,r] = \{z \in \CC_p, |z-e| \leq r\}$. The space $\cA[r](L)$ is a $\QQ_p$-Banach algebra under the norm $||f ||_r = \sup_{e\in\ZZ_p} \sup_n |a_n(e)|r^n$. When $L=\QQ_p$, we simply write $\cA[r]:=\cA[r](\QQ_p)$.



\begin{definition}
Let $\cD[r](\QQ_p)$ denote the Banach-dual $\Hom_{\QQ_p}(\cA[r],\QQ_p)$. It is naturally a $\QQ_p$-Banach module under the norm $||\mu||_r := \sup_{f\in \cA[r]} \frac{ |\mu(f)|}{||f||_r}$. We define $\cD[r](L) :=\cD[r](\QQ_p) \widehat{\otimes} L$.
\end{definition}

\begin{remark}
We note that the notation we have adopted differs from the notation in \cite{HarronPollack}, which defines the above modules as $\Hom_{\QQ_p}(\cA[r],L)$. This latter module is in fact much larger than the spaces we work with, and apparently does not satisfy desired base-change properties. We note that there is an injection $\cD[r](L)\hookrightarrow \Hom_{\QQ_p} (\cA[r], L)$.
\end{remark}

\begin{proposition}[\cite{Bellaiche}, III.4.18 and III.4.21] For $r_1> r_2$, the natural restriction maps $\cA[r_1](L)\rightarrow \cA[r_2](L)$  and $\cD[r_2](L)\rightarrow \cD[r_1](L)$ are injective and compact.
\end{proposition}

\begin{definition}
For any $r\geq 0$, the spaces of \emph{overconvergent} functions and distributions are defined as
\begin{align}
	\cA^\dagger[r](L) &:= \varinjlim_{s>r} \cA[s](L)\\
	\cD^\dagger[r](L) &:=\varprojlim_{s>r} \cD[s](L).
\end{align}
When $r=0$ and $L=\QQ_p$, we simply write $\cA^\dagger = \cA^\dagger[0](\QQ_p)$, $\cD^\dagger = \cD^\dagger[0](\QQ_p)$.
\end{definition}
We will focus primarily on the cases $L=\QQ_p$ and $L=\Lambda = \ZZ_p[[\ZZ_p^\times]]$.

\subsection{Weight $\kappa$ action}
Let $\Sigma_0(p)$ denote the monoid
\begin{equation*}
	\Sigma_0(p) : = \left\{ \begin{pmatrix} a & b \\ c & d\end{pmatrix}\in M_2(\ZZ_p) : a\in \ZZ_p^\times, c\in p\ZZ_p\right\}.
\end{equation*}

\begin{definition}
For each integer $k\geq 0$, we define a (left) weight-$k$ action of $\Sigma_0(p)$ on $\cA^\dagger$ by 
\begin{equation}
	\gamma \cdot_k f(z) := (a+cz)^k f\left(\frac{b+dz}{a+cz}\right).
\end{equation}
This induces a right action on $\cD^\dagger$ by $(\mu|_k\gamma) (f) = \mu(\gamma\cdot_k f)$. We denote by $\cA_k^\dagger$, $\cD_k^\dagger$ the underlying spaces equipped with this weight-$k$ action.
\end{definition}

\begin{remark}
We have a $\Sigma_0(p)$-equivariant surjection $\rho_k : \cD_k^\dagger\rightarrow V_k(\QQ_p)$, called the specialization map, defined by $\rho_k(\mu) = \mu( zX-Y)^k$, which induces a Hecke-equivariant homomorphism $\rho_k : H_c^1(\Gamma, \cD^\dagger_k)\rightarrow H^1_c (\Gamma, V_k(\QQ_p))$. Stevens shows \cite{Stevens} that this homomorphism is in fact an isomorphism for small slope ($<k$) modular symbols. 
\end{remark}

[Replacing the coefficient module $V_k$ by $p$-adic distributions, one obtains an overconvergent Eichler-Shimura isomorphism comparing overconvergent modular symbols with overconvergent modular forms. \cite{Andreatta, Iovita, Stevens}. We probably don't need to say this, but it might motivate the our constructions]



\subsection{Families of Modular Symbols}

\section{Pairings on Coefficient Modules} \label{sec:coeff_pairing}

We have assumed given a $\Gamma$-equivariant pairing $\langle~,~ \rangle : M \times M \longrightarrow E$ (or another trivial $\Gamma$ representation). We explicitly write down the pairing in the cases of interest and check that the necessary equivariance is satisfied. 

\subsection{$\Sym^k(E^2)$}

We identify $\Sym^k(E^2)$ with $E[X,Y]_k$, and endow it with left action of $\GL_2(\QQ)$ by $\gamma \cdot P(X,Y) = P((X,Y)\gamma)=P(aX+cY,bX+dY).$ 
\begin{definition}
Define a bilinear pairing on $\Sym^k(E^2)$ by
\begin{equation*}
	\langle X^iY^{k-i}, X^j Y^{k-j} \rangle = \begin{cases} (-1)^i {k \choose i} & \text{ if } i+j=k\\
												0	& \text{ otherwise} \end{cases}
\end{equation*}
and extending via linearity.
\end{definition}


\begin{lemma}
For all $\gamma\in \GL_2(\QQ)$, $P,Q\in\Sym^k(E^2)$, we have $\langle \gamma\cdot P,\gamma \cdot Q\rangle = \det(\gamma)^k \langle P,Q\rangle$.
\end{lemma}
\begin{proof}
This can be seen via an ugly computation. (There should be an easy way to see this in terms of the wedge product).
\end{proof}

For the purposes of Hecke-equivariance, we define a twisted pairing using the Atkin-Lehner involution $W_N = \begin{pmatrix} 0 & -1 \\ N & 0 \end{pmatrix}$.
\begin{definition}
The twisted pairing on $\Sym^k(E^2)$ is defined by
\begin{equation}
	\langle P, Q\rangle^t := \langle P, W_N \cdot Q\rangle.
\end{equation}
\end{definition}
More explicitly,
\begin{equation*}
	\langle X^iY^{k-i}, X^j Y^{k-j} \rangle_N^t = \langle X^i Y^{k-i}, (-Y)^j(NX)^{k-j}\rangle= \begin{cases} (-1)^k N^i {k \choose i} & \text{ if } i=j\\
												0	& \text{ otherwise} \end{cases}
\end{equation*}

We can view $\Sym^k(\QQ_p^2)$ as a quotient of $\cD_k$, and the pairings from the previous section extend uniquely in a $\Sigma_0(p)$-equivariant way to $\cD_k$. Moreover, the twisted pairing behaves well with respect to our finite approximations. [Insert details]

\begin{definition}
Let $\langle~,~\rangle^t_N : \cD_k\times\cD_k\longrightarrow \QQ_p$ be the $\QQ_p$-bilinear pairing
\begin{equation}
	\langle \mu, \nu \rangle_N^t := \sum_{n=0}^\infty {k \choose n} (Np)^n \mu(z^n)\nu(z^n)
\end{equation}
\end{definition}




\begin{definition}
Let $\langle ~, ~\rangle^t_N :\cD(\Lambda)\times\cD(\Lambda) \longrightarrow \Lambda$ be the $\Lambda$-bilinear pairing
\begin{equation}
	\langle \mu, \nu \rangle_N^t = \sum_{n=0}^\infty {\frac{\log_p(1+pw)}{\log_p(1+p)} \choose n} (Np)^n \mu(z^n)\nu(z^n).
\end{equation}
\end{definition}

\begin{lemma}
The above pairing interpolates the weight-$k$ twisted pairings, and satisfies the following equivariance property: for all $\gamma \in\Gamma_0(p)$, $\mu,\nu\in \cD(\Lambda)$, 
\begin{equation}
	\langle \mu | \gamma, \nu\rangle_N^t  = \langle \mu, \nu| W_{Np}^{-1}\gamma^* W_{Np}\rangle_N^t.
\end{equation}	
\end{lemma}


\section{Existing Pairings on Modular Symbols} \label{sec:existing_pairings}
% Reference discussions of pairings in Bellaiche and Kim's thesis
% Relate to Petersson inner product
The Petersson inner product is a natural Hermitian inner product on $S_{k+2}(\Gamma,\CC)$ defined by:
\begin{equation*}
	(f,g) := \int_{\cH/\Gamma} f(z) \overline{g(z)} y^{k} dx dy
\end{equation*}

[Here we introduce the Rankin product and adjoint $L$-function, and state the relationship between these and Petersson inner product. Roughly $(f,f)=L(\ad f,k+1)$]

The pairing on $V_k$ [where to put this] and the cup product induce a pairing on $H^1_c(V_k)$:
\begin{equation*}
	H^1_c(V_k)\times H^1_c(V_k) \longrightarrow H^2_c(V_k\otimes V_k)\longrightarrow H^2_c(\CC)\cong \CC
\end{equation*}
Moreover, this cup product pairing is compatible with the Petersson inner product.
\begin{proposition}[III.2.6 of Bella\"iche]
\begin{equation*}
	(\phi_f,\phi_g ) = (2i)^{k-1}[(f,g) + (-1)^{k+1} (g,f)]
\end{equation*}
\end{proposition}


[Bella\"iche, section III.2.7]
Hecke operators: Suppose now that $\Gamma_1(N)\subseteq \Gamma \subseteq \Gamma_0(N)$. Let 
\begin{equation}
	W_N = \begin{pmatrix} 0 & -1 \\ N & 0\end{pmatrix}
\end{equation}
which normalizes $\Gamma_0(N)$ and (after some work) $\Gamma$. See Lemma III.2.4.

The ``corrected" scalar product:
\begin{equation*}
	[f,g] : =(f,g|W)
\end{equation*}

\begin{lemma}[III.2.26]
The Hecke operators $T_\ell$ ($\ell\nmid N$, $U_q$ , and $\langle a \rangle$ are self-adjoint with respect to the corrected scalar product:
\begin{equation*}
	[f|T,g]=[f,g|T].
\end{equation*}	
\end{lemma}







\section{Reducing to Level 1} \label{sec:level_one}

Suppose that $\Phi, \Psi \in \Symb_{\Gamma_0(N)}(M)$ are overconvergent modular symbols.  As in Section \ref{sec:mod_curve_cohom}, we associate to them homogeneous $1$-cocycles in $Z^1(\Gamma_0(N), M)$ by the following formulas:
\begin{align*}
\sigma_{\Phi}(\alpha) &= \Phi(\{\infty, \alpha \cdot \infty\}) \\
\sigma_{\Psi}(\beta) &= \Psi(\{\alpha \cdot \infty, \alpha \cdot \beta \cdot \infty\}).
\end{align*}
%f(\alpha) \otimes g(\alpha \cdot \beta)
The pairing $M \times M \to E$ from Section \ref{sec:coeff_pairing} then induces a natural cup product pairing $\Hh^1_P(\Gamma_0(N), M) \times \Hh^1_P(\Gamma_0(N), M) \to \Hh^2_P(\Gamma_0(N), E)$.  In order to compute this pairing, we need to give an explicit isomorphism $\omega : \Hh^2_P(\Gamma_0(N), E) \to E$, which we do in two steps.

First, corestriction provides an isomorphism $\cor : \Hh^2(\Gamma_0(N), E) \to \Hh^2(\SL_2(\ZZ), E)$, given by multiplication by $[\SL_2(\ZZ) : \Gamma_0(N)]$, \todo{need isomorphism on parabolic cohomology, not normal cohomology} so we need only compute an isomorphism $\Hh^2_P(\SL_2(\ZZ), E) \to E$.  Using Theorem \ref{thm:H2E}, we have
\[
\Hh^2_P(\SL_2(\ZZ), E) \cong Z^2(\SL_2(\ZZ), E) / B^2_P(\SL_2(\ZZ), E) \cong B^2(\SL_2(\ZZ), E) / B^2_P(\SL_2(\ZZ), E).
\]
Given a cocycle $\tau \in Z^2(\SL_2(\ZZ), E)$, to compute $\omega([\tau]) \in E$ we thus write $\tau = d\kappa$ for some $\kappa \in C^1(\SL_2(\ZZ), E)$ and set
\[
\omega([\tau]) = \kappa\left( \begin{pmatrix} 1 & 1 \\ 0 & 1 \end{pmatrix}\right).
\]
It remains to explain how to write $\tau = d\kappa$ as a coboundary.

The key fact used in computing $\kappa$ is that $\SL_2(\ZZ)$ is generated by torsion elements
\begin{align*}
R &= \begin{pmatrix} 0 & -1 \\ 1 & -1 \end{pmatrix}, \\
S &= \begin{pmatrix} 0 & -1 \\ 1 & 0 \end{pmatrix};
\end{align*}
note that $R$ has order $3$ and $S^2 = -I$.  We can determine $\kappa(R)$ and $\kappa(S)$ as follows, using the construction of $\kappa$ as $d\kappa = \tau$.
\begin{align*}
\tau(S,S) &= 2\kappa(S) - \kappa(-I) \\
\tau(-I,-I) &= 2\kappa(-I) - \kappa(I) \\
\tau(I,I) &= \kappa(I).
\end{align*}
Moreover, since $\tau$ is defined as a cup product of $\sigma_\Phi$ and $\sigma_\Psi$, which vanish on $\pm I$, we have that $\tau(\pm I, A) = \tau(A, \pm I) = 0$ for any $A \in \SL_2(\ZZ)$.  Thus
\begin{equation}
\kappa(S) = \frac12 \tau(S,S).
\end{equation}
Similarly,
\begin{align*}
\tau(R,R) &= 2\kappa(R) - \kappa(R^2) \\
\tau(R,R^2) &= \kappa(R) + \kappa(R^2),
\end{align*}
and solving this linear system gives
\begin{equation}
\kappa(R) = \frac13 \tau(R,R) + \frac13 \tau(R,R^2).
\end{equation}
Now, an arbitrary element of $\SL_2(\ZZ)$ can be decomposed as a word in $R$ and $S$ using the Euclidean algorithm, say $A = \prod_{i=1}^m x_i$, where each $x_i$ is ether $R$ or $S$.  We use the equations
\begin{align*}
\tau(x_1, \prod_{i=2}^m x_i) &= \kappa(x_1) + \kappa(\prod_{i=2}^m x_i) - \kappa(A) \\
&\hspace{5pt}\vdots \\
\tau(x_{m-1}, x_m) &= \kappa(x_{m-1}) + \kappa(x_m) - \kappa(x_{m-1}x_m)
\end{align*}
to solve for $\kappa(A)$ in terms of $\kappa(R)$, $\kappa(S)$ and values of $\tau$.

\section{Application to Adjoint $L$-functions} \label{sec:adjoint}

[Here we follow Kim's thesis and Bella\"iche's book. Maybe Hida in the ordinary case?]

The main theorem will look like
\begin{theorem}[Prototype]
\begin{equation}
	L^{alg}(\ad f , k-1) = [\Phi_{f}^+,\Phi_{f}^-]
\end{equation}
\end{theorem}
One problem is that $L^{lag}$ is not well-defined---it depends on a choice of complex periods. See Kim, Theorem 3.9.3 for a precise (if suspect!) statement. Note that this only handles the case that $f$ has level $Np^r$ for $r\geq 1$ and non-trivial nebentype. Will need Hida for a reference in the ordinary case, since Bella\"iche is incomplete.

Note: Kim's results seem to require a Gorenstein property of the Hecke algebra. Bella\'iche's construction of $L^{adj}$ is slightly different: $L^{adj}$ is defined as a generator of a canonical ideal $\mathcal{L}^{adj}$ whenever that ideal is principal. A precise relation to Kim's definition (and ours) is given by Bella\"iche Proposition VI.4.5.

\begin{definition}
A closed point $x\in C^o$ with field of definition $L$ is a \emph{good point} if [Insert long definition here?]
Bella\ $\Symb_\Gamma(\cD(L))^\pm[x]$ are $1$-dimensional.
\end{definition}

\begin{proposition}
Let $L$ be a finite extension of $\QQ_p$ and let $x\in \mathcal{C}^o(L)$ be a good point [see REF]. Fix $U=\Sp\mathcal{T}$ a sufficiently small clean neighborhood of $x$ [see REF]. Then for every sufficiently small real $r>0$ there exist two real constants $0<c<C$ such that for every point $y\in U$ with $\kappa(y)=w$ one has
\begin{equation}
	c| [\Phi_{y}^+, \Phi_y^-]_w | \leq |L^{adj}(y) | \leq C | [\Phi_y^+, \Phi_y^-]_w|,
\end{equation}
where $\Phi_{y}^\pm$ are generators of the spaces $H^1_p(\Gamma,\cD_w[r](L))^{\pm}[y]$ such that $||\Phi_y^\pm||_r = 1$. Moreover, the adjoint $p$-adic $L$-function is characterized (up to an element of $\mathcal{T}^\times$) by the above property.
\end{proposition}

\begin{corollary}
On a small clean neighborhood of $x$, the zero loci of $L^{alg}$ and $L^{adj}$ coincide.
\end{corollary}

\begin{remark}
The above corollary probably extends to the entire ordinary part of the cuspidal eigencurve. See Hida's book or ``Iwasawa modules of congruences of cusp forms".
\end{remark}

\end{document}