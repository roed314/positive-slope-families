\documentclass[10pt]{amsart}

\title{Computing Adjoint $L$-functions}
\author{David Roe}
\address{Department of Mathematics, University of Pittsburgh, 301 Thackeray Hall, Pittsburgh, PA , United States, 15260.}
\email{roed.math@gmail.com}
\author{Ander Steele}
\address{Department of Mathematics, University of Calfornia Santa Cruz, ADD ADDRESS HERE.}
\email{gasteele@ucsc.edu}

%\subjclass[2010]{}
%\keywords{}

\usepackage{amsmath}
\usepackage{amssymb}
\usepackage{amsrefs}
% Fonts
\usepackage{mathrsfs}
% Enumitem
\usepackage{enumitem}
% Hyperrefs
\usepackage{hyperref}

%\usepackage{tikz}
%\usetikzlibrary{shapes,arrows,calc,matrix}
%\usepackage{tikz-cd}

%%%%% for spacing
\usepackage{lipsum}
\usepackage{setspace}

%%%%%%%%%%%%%%% THEOREM STYLES %%%%%%%%%%%%%%%
\theoremstyle{plain}
      \newtheorem{theorem}{Theorem}[section]
      \newtheorem*{theorem*}{Theorem}
      \newtheorem{proposition}[theorem]{Proposition}
      \newtheorem{lemma}[theorem]{Lemma}
      \newtheorem{corollary}[theorem]{Corollary}

      \theoremstyle{definition}
      \newtheorem{definition}[theorem]{Definition}

      %\theoremstyle{remark}
      \newtheorem{remark}[theorem]{Remark}
      \newtheorem{example}[theorem]{Example}
      
      \newtheorem{conjecture}[theorem]{Conjecture}
%%%%%%%%%%%%%%% RINGS AND GROUPS %%%%%%%%%%%%%%%
\newcommand{\ZZ}{{\mathbb{Z}}}
\newcommand{\NN}{{\mathbb{N}}}
\newcommand{\CC}{{\mathbb{C}}}
\newcommand{\QQ}{{\mathbb{Q}}}
\newcommand{\RR}{{\mathbb{R}}}
\newcommand{\OK}{\mathcal{O}_K}
%%%%%%%%%%%%%%% NAMED OPERATORS %%%%%%%%%%%%%%%
\DeclareMathOperator{\Gal}{Gal}
\DeclareMathOperator{\Hom}{Hom}
\DeclareMathOperator{\ord}{ord}

\DeclareMathOperator{\GL}{GL}
\DeclareMathOperator{\SL}{SL}
\DeclareMathOperator{\PGL}{PGL}
\DeclareMathOperator{\Sym}{Sym}
\DeclareMathOperator{\Symb}{Symb}


%%%%%%%%%%%% BEGIN DOCUMENT %%%%%%%%%%%
\usepackage{todonotes}

\begin{document}
\maketitle

\section{Introduction}

\section{Review}

Let $E$ be a complete, discrete valued field.  \todo{think about what works in characteristic $p$}

\section{Cohomology of Modular Curves}
% Reference appendix to Hida's Elementary Theory of L-functions and Eisenstein Series

\section{Existing Pairings on Modular Symbols}
% Reference discussions of pairings in Bellaiche and Kim's thesis
% Relate to Petersson inner product

\section{Pairings on Coefficient Modules}

We have assumed given a $\Gamma$-equivariant pairing $\langle~,~ \rangle : M \times M \longrightarrow E$ (or another trial $\Gamma$ representation). We explicitly write down the pairing in the cases of interest and check that the necessary equivariance is satisfied. 

\subsection{$\Sym^k(E^2)$}

We identify $\Sym^k(E^2)$ with $E[X,Y]_k$, and endow it with left action of $\GL_2(\QQ)$ by $\gamma \cdot P(X,Y) = P((X,Y)\gamma)=P(aX+cY,bX+dY).$ 
\begin{definition}
Define a bilinear pairing on $\Sym^k(E^2)$ by
\begin{equation*}
	\langle X^iY^{k-i}, X^j Y^{k-j} \rangle = \begin{cases} (-1)^i {k \choose i} & \text{ if } i+j=k\\
												0	& \text{ otherwise} \end{cases}
\end{equation*}
and extending via linearity.
\end{definition}


\begin{lemma}
For all $\gamma\in \GL_2(\QQ)$, $P,Q\in\Sym^k(E^2)$, we have $\langle \gamma\cdot P,\gamma \cdot Q\rangle = \det(\gamma)^k \langle P,Q\rangle$.
\end{lemma}
\begin{proof}
This can be seen via an ugly computation. (There should be an easy way to see this in terms of the wedge product).
\end{proof}

For the purposes of Hecke-equivariance, we define a twisted pairing using the Atkin-Lehner involution $W_N = \begin{pmatrix} 0 & -1 \\ N & 0 \end{pmatrix}$.
\begin{definition}
The twisted pairing on $\Sym^k(E^2)$ is defined by
\begin{equation}
	\langle P, Q\rangle^t := \langle P, W_N \cdot Q\rangle.
\end{equation}
\end{definition}
More explicitly,
\begin{equation*}
	\langle X^iY^{k-i}, X^j Y^{k-j} \rangle_N^t = \langle X^i Y^{k-i}, (-Y)^j(NX)^{k-j}\rangle= \begin{cases} (-1)^k N^i {k \choose i} & \text{ if } i=j\\
												0	& \text{ otherwise} \end{cases}
\end{equation*}

\section{Reducing to Level 1}

\section{Application to Adjoint $L$-functions}

\end{document}